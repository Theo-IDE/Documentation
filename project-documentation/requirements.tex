\startcomponent[requirements]

\startchapter[title={Anforderungsanalyse}, reference={chapter:requirements}][author={Peter Preinesberger}]
Die Anforderungen an die zu entwickelnde Lösung sind seitens des Auftraggebers mit Absicht nur grob definiert.
Bis auf die Tatsache, dass man mit dem Projektergebnis Software in den Sprachen ausführen und analysieren
können sollte, gibt es im wesentlichen keine rigiden Vorgaben. Allerdings hat der Auftraggeber Vorschläge
für technische Frameworks und die Gestalt der Lösung angebracht, welche von uns als Anforderungen übernommen
wurden. Dieser Abschnitt beinhaltet die Beschreibung von diesen Anforderungen, sowie weitere funktionale und
nicht-funktionale Bedingungen die sich aus dem Projektkontext ergeben und im Team als Ziel definiert wurden.

\startsection[title={Technische Anforderungen}, reference={section:technical-requirements}]
% die "Vorschläge", also CMake, QT, C++, Bytecode Compiler + VM
\stopsection

\startsection[title={Funktionale Anforderungen}, reference={section:functional-requirements}]
\stopsection

\startsection[title={Nicht||Funktionale Anforderungen}, reference={section:non-functional-requirements}]
\stopsection

\stopchapter

\stopcomponent
