\startenvironment[env_project_documentation]
\setuppapersize[A4][A4]
\language[de]
\mainlanguage[de]
\setuplanguage[de][lefthyphenmin=2]
\setuphyphenmark[sign=normal]
\setuplayout[width=middle,height=27cm,backspace=2.5cm,topspace=1.5cm,header=1em,headerdistance=1cm,footer=1em,footerdistance=1cm]
\setupexternalfigures[directory={graphics},location=global]
\setuppagenumbering[location={footer,right}]
\usepath[{typescripts}]
\usetypescriptfile[scientific-collection]
\usetypescript[scientific-collection]
\setupbodyfont[scientific-collection,rm,12pt]
\setuptyping[numbering=line,lines=no,space=normal,strip=yes,keeptogether=yes,style=\ttx]
\setuplinenumbering[typing][style=\ttx]
\setupcolors[state=start]
\usecolors[crayola]
\definecolor[AccentColor][r=0.6510,g=0.7529,b=0.2510]
\definecolor[AccentColorDark][r=0.4745098039,g=0.4549019608,b=0.0549019608]

\definedescription[detaildescription][
    headstyle=bold, style=normal, align=flushleft, alternative=hanging, 
    width=broad, margin=yes]

\define[2]\ChapterHeadStyle{%
  \startframed[frame=off,align=flushleft]
  {\tfa#1\crlf}
  {\tfd#2}
  \stopframed}

\def\PlaceAuthor#1{%
  \noindent\setupinterlinespace[small]\color[CharcoalGray]{\tfx\it von #1}}

\def\AuthorCommand{%
  {\doifsomething 
    {\structureuservariable{author}}
    {\PlaceAuthor{\structureuservariable{author}}\blank[small]\nobreak}}}

\setuphead[chapter][command=\ChapterHeadStyle,header=high,style=\tf,numberstyle=sc,numbercolor=AccentColorDark,insidesection=\AuthorCommand,after=\nobreak]
\setuphead[section][style=\tfc,tolerance=tolerant,insidesection=\AuthorCommand,after=\nobreak]
\setuphead[subsection][style=\tfb]
\setuphead[subsubsection][style=\tfa]
\setuphead[title][style=\tfd,command=]
\setuphead[subject][style=\tfc]
\setuphead[subsubject][style=\tfb]
\setuphead[subsubsubject][style=\tfa]

\setuplabeltext[de][chapter=Abschnitt~]

\setuplayouttext[header][text]
  [righttext={\getmarking[chapter]},
    lefttext={\ifx\equal{\expanded\getmarking[chapternumber]}{}
      {}
      \else
      {\labeltext{chapter} \getmarking[chapternumber]}
      \fi}]

\setupwhitespace[medium]

\setupinteraction
  [state=start,
    title={Theo-IDE},
    author={Theo-IDE Development Team},
    subtitle={Projektdokumentation},
    keyword={code,green,development,project},
    color=MidnightBlue,
    contrastcolor=TealBlue]
\setupinteractionscreen[option=bookmark]

\startuniqueMPgraphic{coverpage}
StartPage;
  numeric w; w = PaperWidth-eps;
  numeric h; h = PaperHeight-eps;
  numeric line_h; line_h = 0.7h;
  picture headline; headline := image (draw rawtextext("\framed[align=flushleft,frame=off,offset=none]{\definedfont[SansBold]Entwicklungsumgebung und Compiler\crlf für LOOP, WHILE und GOTO}"));
  picture subhead; subhead := image (draw rawtextext("\definedfont[Sans]Projektdokumentation"));
  picture names; names := image(
    draw rawtextext("\framed[align=flushright,frame=off]{\definedfont[Sans]Florian Korbacher\crlf Peter Preinesberger\crlf Kevin Fick}")
  );
  headline := headline xsized (w - 2cm);
  subhead := subhead xsized(0.5w);
  names := names xsized(.3w);
  draw headline shifted (1cm,line_h + 5mm);
  draw subhead shifted ((1cm,line_h - 5mm) - ulcorner subhead);
  draw names shifted (lrcorner Page - lrcorner names - (1cm, -1cm));
  draw (0,line_h)--(w,line_h) withpen pencircle scaled 1mm withcolor \MPcolor{AccentColor};
StopPage;
setbounds currentpicture to Page;
\stopuniqueMPgraphic



\defineoverlay[coverpage][\uniqueMPgraphic{coverpage}]

\setupTABLE[loffset=5pt,roffset=5pt,toffset=2.5pt,boffset=2.5pt]
\setupTABLE[each][align={flushleft,lohi}]

\definehighlight[Code][style=\tt]
\definehighlight[Important][style=\it]
\definehighlight[EnumHead][style=\bf]

\definereferenceformat[infig][text=\labeltext{figure}]
\definereferenceformat[intable][text=\labeltext{table}]
\definereferenceformat[insection][text=\labeltext{section}]
\definereferenceformat[inchapter][text=\labeltext{chapter}]

\def\asin[#1:#2]{%
  \expandafter\ifx\csname in#1\endcsname\relax
    \writestatus{warning}{referenceformat in#1 not defined}%
    \in[#1:#2]%
  \else
    \csname in#1\endcsname[#1:#2]%
  \fi}

\defineframedtext
  [framedcode]
  [strut=yes,
   offset=5pt,
   width=\textwidth,
   location=middle,
   corner=rectangular,
   background=color,
   backgroundcolor=CulturedPearl,
   align=right,
   frame=off]
   
\definetyping[command][numbering=none,
                    before={\startframedcode},
                    after={\stopframedcode}]

\stopenvironment
