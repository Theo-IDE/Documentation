\startcomponent[introduction]

\startchapter[title={Einleitung},reference={chapter:introduction}][author={Peter Preinesberger}]
Das Projekt \Important{Entwicklungsumgebung und Compiler für LOOP, WHILE und
GOTO} entstand im Rahmen der Projektarbeit des Bachelorstudiengangs Informatik
an der Technischen Hochschule Würzburg-Schweinfurt. Konzipiert als
Hilfestellung für Studenten des Moduls \Important{ Grundlagen der Theoretischen
Informatik} enthält das Projekt Software, die das Auseinandersetzen mit
Konzeptsprachen wie LOOP, WHILE und
GOTO\footnote{\goto{https://de.wikipedia.org/wiki/WHILE-Programm}[url(https://de.wikipedia.org/wiki/WHILE-Programm)]}
erleichtern soll. Die entwickelten Module bieten Möglichkeiten zur Kompilation,
Interpretation und dem Debuggen von Programmen, welche in einer Sprache
verfasst sind, die so nah wie möglich an den erwähnten Konzeptsprachen gehalten
ist. Das Projekt beinhaltet hierbei Benutzerschnittstellen für die
Kommandozeile, als auch eine intuitive grafische Oberfläche.

\startsection[title={Zweck dieser Dokumentation},reference={section:structure}][author={Peter Preinesberger}]
Dieses Dokument dient der Zusammenfassung der Anforderungen, der Dokumentation
und Erläuterung von Entscheidungen während des Entwicklungsprozesses und der
allgemeinen technischen Aufarbeitung der Lösung. Zielgruppe dieses Dokuments
sind der Auftraggeber, Prof. Dr. Frank Deinzer, sowie weitere Beiwohnende der
Prüfung, Entwickler und alle, die an den technischen Details oder am Hergang
der Entwicklung dieses Projekts interessiert sind. \Important{Ein
Benutzerhandbuch, welches sich an die geeignete Zielgruppe wendet, wird in
einem separaten Dokument zur Verfügung gestellt}.

\stopsection

\startsection[title={Aufbau der Dokumentation},reference={section:structure}]

\stopsection

\stopchapter
\stopcomponent
